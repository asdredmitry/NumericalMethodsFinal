\documentclass[titlepage]{article}
\usepackage[12pt]{extsizes} 
\usepackage[T2A]{fontenc}
\usepackage[utf8]{inputenc}
\usepackage[english,russian]{babel}
%\usepackage{pscyr}
\usepackage{hyperref}
\usepackage{setspace}
\usepackage{amsmath,amssymb,amsfonts,amsthm,secdot}
\usepackage[left=30mm, top=20mm, right=30mm, bottom=20mm, nohead, footskip=15mm]{geometry} 
\usepackage[pdftex]{graphicx}
\usepackage[indentfirst]{titlesec}
\usepackage[usenames]{color}
\usepackage{colortbl}
\usepackage{listings}
\usepackage{secdot}

\def\l{\left}
\def\r{\right}
\def\le{\leqslant}
\def\ge{\geqslant}

\begin{document} 

\newtheorem{theorem}{Теорема}
\newtheorem{lemma}{Лемма}
\newtheorem{definition}{Определение}
\renewcommand{\proofname}{Доказательство}

\begin{center}
\hfill \break
\hfill \break
\hfill \break
\LARGE Вычисление несобственного интеграла с помощью квадратурных формул \\
\hfill \break
\large Д.А. Михайлин \\
\hfill \break
\today \\

\end{center}

\section{Постановка задачи}
Требуется приближенно вычислить интеграл: 
\begin{gather}
	\int_{0}^{+\infty}(x + \frac{1}{x})\frac{(sin{3x})^2}{\ln(1 + x)}e^{-x^2}dx
\end{gather}
с заданной точностью $\varepsilon = 10^{-2}$.

\section{Доказательство сходимости интеграла}
Прежде чем приступить к вычислению интеграла, необходимо убедиться в его сходимости во всей области интегрирования. Для этого исследуем подынтегральную функцию $f(x) = (x + \frac{1}{x})\frac{(sin{3x})^2}{ln(1 + x)}e^{-x^2}$ в особых точках на промежутке $[0, +\infty)$. Рассмотрим две особые точки: $x = 0$,  $x \to \infty$:

\begin{enumerate}
	\item $x = 0$ \\
	\begin{gather}
		\notag \lim_{x \to 0}{f(x)} = \lim_{x \to 0}{(x + \frac{1}{x})\frac{(sin{3x})^2}{\ln(1 + x)}e^{-x^2}} = \\
		\notag = \lim_{x \to 0}{\frac{9x^2 + o(x^2)}{x(x + o(x))}(1 + x^2)(e^{-x^2})} = 9
	\end{gather}
	Таким образом, $x = 0$ является устранимой особой точкой. Это значит, что в нуле мы можем доопределить $f(x)$ до непрерывной функции, поэтому интеграл не является несобственным в точке $x = 0$.
\item $x \to \infty$ \\
	
	$\exists$ такое большое число $C$, что:
	\begin{gather}
		\notag \l| \int_{C}^{+\infty}{f(x)dx} \r| \le \int_{C}^{+\infty}{\l| \frac{x + 1}{x^2}\frac{sin^2{3x}}{\ln{(1+x)}}e^{-x^2}\r|dx^2} \le \\ \notag \le\int_{C}^{+\infty}{ 2e^{-x^2} dx^2} = 2e^{-C^2} < \infty
	\end{gather}
Сходимость интеграла на бесконечности доказана.
\end{enumerate}
\section{Оценки параметров $\delta_1$, $C$}
Положим $\varepsilon_1 = \varepsilon_2 = 2\cdot10^{-3}$ $\varepsilon_3 = 6\cdot10^{-3}$ и разобьем наш промежуток интегрирования на 3 части: $[0, \delta_1), [\delta_1, C), [C, +\infty)$.

Теперь нам необходимо подобрать числа $\delta_1$ и $C$ таким образом, чтобы $\l|\int_{0}^{\delta_1}{f(x)dx}\r| \le \varepsilon_1$ и $\l|\int_{C}^{+\infty}{f(x)dx}\r| \le \varepsilon_3$. Тем самым, задача сведется к вычислению интеграла $\tilde I_2 = \int_{\delta_1}^{С}{f(x)dx}$ с точностью $\varepsilon_2$.	
\begin{enumerate}
	\item Оценим $\delta_1$. \\
	Будем пользоваться следующими фактами:

	\begin{gather}
		\notag \sin(x) \le x  \\ 
		\notag \text{и} \\
		\notag \frac{x}{x + 1} \le \ln{(x + 1)}
	\end{gather}

	\begin{gather}
		\notag \l|\int_{0}^{\delta_1}{f(x)dx}\r| \le \int_{0}^{\delta_1}{\l|f(x)\r|dx} \le \int_{0}^{\delta_1}{\l| (x + \frac{1}{x})\frac{(sin{3x})^2}{\ln(1 + x)}e^{-x^2}\r| dx} \le \\
		\notag \le \int_{0}^{\delta_1}{\l| \frac{x^2 + 1}{x}e^{-x^2}\frac{x + 1}{x}9x^2\r|dx} 
		\notag \le \int_{0}^{\delta_1}{18 dx} = 18\delta_1 \le \varepsilon_1
	\end{gather}
	Следовательно, $\delta_1 \le \varepsilon_1/18$
	Подставив $\varepsilon_1 = 2\cdot10^{-3}$, получим $\bf{\delta_1 = 0.0001}$.
	\item Оценим $C$.
		\begin{gather}
		\notag \l|\int_{C}^{+\infty}{f(x)dx}\r| \le \int_{C}^{+\infty}{\l| (x + \frac{1}{x})\frac{(sin{3x})^2}{\ln(1 + x)}e^{-x^2}\r|dx} \le \\
		\notag \le \int_{C}^{+\infty}{\l|(\frac{x^2 + 1}{x})\frac{(sin{3x})^2}{\ln(1 + x)}e^{-x^2}\r|dx^2} \le \frac{C^2 + 1}{2C^2}\frac{1}{\ln{(1 + C)}}\int_{C}^{+\infty}{ e^{-x^2} dx^2} = \frac{C^2 + 1}{2C^2}\frac{1}{\ln{(1 + C)}}e^{-C^2} = \\F(C) \le \varepsilon_3 \\
	F(2.1) > 0.006\\
	F(2.123) = 0.0059
	\end{gather}
	Возьмем C = 2.123.
\end{enumerate}
\section{Квадратурные формулы для $I_2$}
Будем вычислять $I_2$ по составной формуле прямоугольников. Для функции $f(x)$, определенной на некотором отрезке $[a,b]$ интеграл от нее приближенно вычисляется по формуле:
$$I(f) = \int_{a}^{b}{f(x)dx} \approx S_N(f) = \frac{1}{N}\sum_{i=1}^{N}{f\l(a+(b-a)\frac{i-1/2}{N}\r)}$$
\section{Оценка погрешности квадратурных формул}
Погрешностью квадратурной формулы является величина:
$$R_N(f) = I(f) - S_N(f)$$
Для формулы прямоугольников справедлива следующая оценка:
$$\l|R_N(f)\r| \le A\frac{(b-a)^2}{4N},$$
для некоторого числа $A$, такого, что $\l|f'(x)\r| \le A$ на $[a,b]$. Таким образом, для вычисления необходимого числа отрезков разбиения нам необходимо оценить $\l|f'(x)\r|$ на $[a,b]$.
\section{Оценка производной}
Рассмотрим производную функции $f(x)$.
Хотим оценить ее на отрезке $[\delta_1, 2.123]$
Заметим, что $\l|(uvg)'\r| \le \l|u'vg\r| + \l|uv'g\r| + \l|uvg'\r|$.
Разобъем функцию $f(x) = (x + \frac{1}{x})\frac{(sin{3x})^2}{\ln(1 + x)}e^{-x^2}$ на $u = e^{-x^2}$ $v = 
(x + \frac{1}{x})sin(3x)$ $g = \frac{sin(3x)}{ln(1 + x)}$
\begin{gather}
		\notag \l|e^{-x^2}\r| \le 1 \\
		\l|2xe^{-x^2}\r| \le 4.25
		\
\end{gather}
Воспользуемя разложением синуса и косинуса в ряд тейлора с остаточным членом в форме лагранджа. $sin(x) = x - \frac{sin(c)}{2}x^2$ и $cos(x) = 1 - sin(c)x$
\begin{gather}
	\notag \l|v\r| = \l|(x + 1/x)*(3x - \frac{sin(c)9x^2}{2})\r| \le \l|(x^2 + 1)(3 + 4.5x^2)\r| = 128.217 
\end{gather}
\begin{gather}
	\notag \l|v'\r| =\l|(1 - \frac{1}{x^2})sin(3x) + 3(x + \frac{1}{x})cos(3x)\r| \\ \le \l|6x - sin(c)4.5x^2 + sin(c)4.5 - cos(c)3x^2 -cos(c)3 \r| \le 66.2
\end{gather}
\begin{gather}
	\notag \l|g\r| \le \l|(3 - sin(c)4.5x)(x + 1)\r| \le 39.204
\end{gather}
Воспользуемя разложением в ряд тейлора с остаточным членом в форме лагранджа.
$ln(1 + x) = x - \frac{1}{2(1 + c)^2}x^2$
\begin{gather}
	\notag \l|g'\r| = \l|\frac{3ln(1 + x)cos(3x) - \frac{sin(3x)}{1 + x}}{ln^2{(1 + x)}}\r| \le\\ \l|\frac{3(x - \frac{x^2}{2(1 + c)^2})(1 - 3sin(c)x)(1 + x)^2 - (1 + x)(3x - 4.5sin(c)x^2)}{x^2}\r| \le \\ \le 225.007
\end{gather}
Следовательно $|f'(x)| \le 52808.159208 $ 
\section{Таблица параметров}
Вычислим $N_1$:
$$N_1 = \frac{A_1(2.123-\delta_1)^2}{4\varepsilon_2} = 29748846$$

Запишем полученные выше результаты в таблицу:
\begin{center}
	\begin{tabular}{|c|c|c|c|}
		\hline
		$\delta_1$ &  $C$ & $A$ &  $N_1$  \\
		\hline
		0.0001 & 2.123 & 52808.159208 & 29748846 \\
		\hline
	\end{tabular}
\end{center}
\section{Результаты расчёта}
\begin{center}
	\begin{tabular}{|c|c|}
		\hline
		$\tilde I_2$  & $\tilde I$\\
		\hline
		4.86966 & 4.86966 \\
		\hline
	\end{tabular}
\end{center}

\section{Правило Рунге}
Пусть $I$ - точное значение интеграла, а $S_N$ - его приближенное значение, вычисленное с использованием $N$ обращений к подынтегральной функции. Предположим, что известен главный член погрешности квадратурной формулы:
$$I - S_N = CN^{-m} + o(N^{-m}),$$
где $m$ - известно, а $C$ - нет. Тогда, подставляя в формулы выше $N_1$ и $N_2 = 2N_1$, получим:
\begin{gather}
	\notag I - S_{N_1} = CN_1^{-m} + o(N_1^{-m}) \\
	\notag I - S_{N_2} = CN_2^{-m} + o(N_2^{-m}) \\
	\notag S_{N_1} - S_{N_2} \approx C \frac{N_2^m - N_1^m}{N_2^m N_1^m} \Rightarrow \\
	\notag C \approx \frac{S_{N_1} - S_{N_2}}{N_2^m - N_1^m}N_2^m N_1^m \Rightarrow \\
	\notag I - S_{N_2} = \frac{S_{N_1} - S_{N_2}}{N_2^m - N_1^m}N_1^m
\end{gather}
Тогда, если взять $N_2 = 2N_1 = 2N$ получим:
$$\l| I - S_{2N} \r| = \frac{\l| S_{2N} - S_N \r|}{2^m - 1} \le \varepsilon$$

\section{Результаты по правилу Рунге}
\begin{center}
	\begin{tabular}{|c|c|}
		\hline
		$\hat N_1$ & $\hat I_2$  \\
		\hline
		512 & 4.85199  \\
		\hline
	\end{tabular}
\end{center}


\end{document}